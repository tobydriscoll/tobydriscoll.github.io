\documentclass[11pt]{article}

\usepackage{fullpage}
\usepackage{amsmath}

\begin{document}

\begin{center}
  \bf Exercises
\end{center}

\begin{enumerate}
\item Given polynomials \texttt{p} and \texttt{q}, defined in the
  standard MATLAB way, how do you find their product? Write a
  code for it.
\item Using the |roots| command, find the roots of 2,000 random polynomials
  of degree 4. (Here we mean polynomials whose coefficients are
  independently normally distributed.) Plot them all as dots on a
  single graph in the complex plane. You'll want to limit the view to
  $|x|\le 10$, $|y| \le 10$. 
\item NIST maintains a library of Statistical Reference Datasets. Find
  them on the web, look under ``Linear Regression,'' and download the
  Norris dataset. Import it into MATLAB (editing the downloaded file
  if necessary) and perform a linear least-squares fit. Plot the data
  and fit.
\item Kepler's third law of planetary motion posits the power law
  $T=\alpha R^\gamma$, where $T$ is orbital period and $R$ is
  (maximum) orbital radius. Use a best linear fit to solar system data
  to discover reasonable values of $\alpha$ and $\gamma$. Plot the
  relevant data and fit.
\end{enumerate}


\end{document}
