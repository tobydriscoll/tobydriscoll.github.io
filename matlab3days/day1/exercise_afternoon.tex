\documentclass[11pt]{article}

\usepackage{fullpage}
\usepackage{amsmath}

\begin{document}

\begin{center}
  \bf Exercise
\end{center}

Let $f(x)=\sin(e^x)$ on $-1\le x \le 3$. Define a vector of
  4,000 points in the interval that you will use to measure error. For
  each $n=20,40,80,160,320,640$, let $S_n(x)$ be a cubic spline interpolant
  of data obtained by sampling $f$ at $n$ equally spaced points in the
  interval. Compute the errors $E_n = \max |S_n(x)-f(x)|$ (the max is
  taken over your 4,000 point vector). Using a least squares linear
  fit, hypothesize on the behavior of $E_n$ as a function of $n$. 

\begin{center}
  \bf Challenge: Analyze infant blood pressure data
\end{center}

You will be given a link to a CSV (comma separated value) file with three columns. The first column is elapsed time in milliseconds. The third column is a blood pressure reading taken from an infant. We won't use the second column.

Get as far as you can on the following tasks. If you get through them
all, pull out a cigar and put your feet up on the desk. 

\begin{enumerate}
\item Import the data into MATLAB.
\item Make a plot of the blood pressure as a function of time, for about half a minute of real time. 
\item Find all the local minima of the blood pressure function.
\item Add the minima points as dots in your graph from \#2. Each
  interval between minima corresponds to one heartbeat. 
\item Use cubic
  interpolation to resample the curve at 101 points in a heartbeat,
  and calculate the area under the curve using the trapezoid
  rule. (This quantity relates to the volume of blood pumped during
  the heartbeat.) 
\item Repeat \#5 for every heartbeat. Make a histogram of the values.
\end{enumerate}

Some useful commands: diff, find, for, hist, importdata, interp1, plot, trapz, xlim 

\end{document}
