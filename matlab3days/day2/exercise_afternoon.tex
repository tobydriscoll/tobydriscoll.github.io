\documentclass[11pt]{article}

\usepackage{fullpage}
\usepackage{amsmath}

\begin{document}

\begin{center}
  \bf Exercises
\end{center}

\begin{enumerate}
\item Play the ``chaos game.'' In this game, let $x=[0,0]^T$ and then
  repeatedly replace $x$ by $Ax+b$, where $A=\displaystyle
  \begin{bmatrix}
    1/2&0\\0&1/2
  \end{bmatrix}$, and $b$ is chosen randomly from the three vectors
\[
\begin{bmatrix}
  0\\0
\end{bmatrix}, \qquad
\begin{bmatrix}
  1/2\\0
\end{bmatrix},\qquad
\begin{bmatrix}
  1/4\\ \sqrt{3}/4
\end{bmatrix}.
\]
Add a dot for each value of $x$ into a plot in the $x_1$--$x_2$
plane.
\item Let 
\[
A=
\begin{bmatrix}
  1&2&3\\4&5&6\\7&8&9
\end{bmatrix}, \qquad b=
\begin{bmatrix}
  1\\3\\5
\end{bmatrix}.
\]

\begin{enumerate}
\item Do elimination manually in MATLAB for this system (as in the
  demo). What does the result tell you about the linear system?
\item Use backslash to solve the original linear system. Is the result valid?
\end{enumerate}

\item The eigenvalues of Toeplitz matrices are interesting.
  \begin{enumerate}
  \item Let $A$ be the $10\times 10$ matrix
    \[
    \begin{bmatrix}
      1 & -1 & 0 & \cdots & 0 \\
      -2 & 1 & -1 & \cdots & 0 \\
      & \ddots & \ddots & \ddots & \\
      0 & 0 & \cdots & -2 & 1 
    \end{bmatrix}
    \].
    Plot its eigenvalues as crosses (\texttt{'x'}) in the complex plane.
  \item Repeat (a) for the same matrix at size 50, 100, and
    200. Superimpose all the plots.
  \item Repeat (a)--(b) for the Toeplitz matrix whose first row starts
    with $4,-1,1$, and whose first column starts with $4,-2,1$. 
 \end{enumerate}

\end{enumerate}

\end{document}
