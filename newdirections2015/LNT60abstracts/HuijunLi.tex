 \documentclass[11pt]{article}
 \usepackage{latexsym}
 \usepackage{amssymb,amsmath}

 \topmargin 0mm
 \oddsidemargin 5mm
 \evensidemargin 5mm
 \textwidth 150mm
 \textheight 662.801 pt

 \begin{document}

 \begin{center}
   {\sf ~\\[14pt]
 Removing of the Cauchy Singularity in Hilbert Transform Over Unit Circle and Its High Accuracy Quadrature%
     }
 \end{center}
 %%%%%% Names of authors and their addresses: %%%%%%
 \footnotesize{
   \begin{center}
Li, Huijun$^a$, Chongyin Li$^{a, b}$, Xiaobing Pan$^a$, and Yingying Huang$^{a, c}$\\[14pt]

$^a$College of Meteorology and Oceangraphy, PLA University of Science and Tecnology, Nanjing {\rm 211101}, China;\\[3mm]

$^b$State Key Laboratory of Numerical Modeling for Atmospheric Science and Geopysical Dynamics, Institute of Atmospheric Physics, Chinese Academy of Sciences, Beijing {\rm 100029}, China;\\[3mm]

$^c$School of Electronic Information, Wuhan University, Wuhan, 430079, China;
   \end{center}
   }
 %%%%%% (Obvious tweaks for a different number of authors.)


 \normalsize
 \noindent
Hilbert transform over unit circle has wide applications in study of aerodynamics, atmospheric gravity waves, and plasma physics of the polar aurora, {\it etc}.  It has been traditionally solved with the Method of Discrete Vortices (MDV),  and recently, MDV has also been proved to be a method of point-wise super convergency. In order to construct the new quadrature scheme with high accuracy, we present a transform to the Hilbert transform formulae in this talk, in which the Cauchy singularities of Hilbert transform are removed, and thus these new formulae are smooth enough to be integrated with the {\it chebfun} toolbox. In order to deal with the ill-posed linear operators appearing in solving the the inverse boundary value problems (IBVPs) of Laplace’s equation on a circle, we construct an iterated-Tikhonov type regularization scheme, where the coefficient matrix are constructed the the {\it chebfun} tool. We focus on relations between the spectrum  accuracy of the Chebfun series and the ill-posedness of the IBVPs.  Accuracy and efficiency of our Bench-testing results in contrast to our previous study with the MDV are also presented here. Lastly, we discuss an idea about extension of this new Hilbert transform formula over unit circle through a conformal mapping procedure to any plane region, such as the upper half plane, the plane rectangle, and the crescent region. It is hardly to find a MDV analogue that keeping the super convergency in these new plane regions, however,  whenever we extent this new Hilbert transform formula to them, the high accuracy is kept naturally with the help of {\it chebfun}. It is believed that characteristics of  high accuracy and high encapsulation of  {\it chebfun} should take great conveniency in our study on structure of flux ropes, gravity waves, and the aurora phenomena.
\end{document}

%%%%%%%%%%%%%%%%%%%%%%%%%%%%%%%%%%%%%%%%%%%%%%%%%%%%%%%%%%%%%%%%%%%
